\documentclass{article}

\usepackage[charter]{mathdesign}
\usepackage[italian]{babel}

\usepackage{mathtools}
\usepackage{amsmath}
\usepackage{esint}

\title{OTTICA DIFFRATTIVA}
\author{}
\date{}

\begin{document}
\maketitle

Lavoriamo sempre con le equazioni di Maxwell facendo un analisi in frequenza.
\begin{equation}
\textbf{E}(\textbf{r}, t)= \textbf{E}(\textbf{r}) e^{-i \omega t}
\end{equation}

\begin{equation}
\nabla^2 \textbf{E} (\textbf{r}) + \left(\frac{\omega}{c}\right)^2 \frac{2}{\varepsilon} \textbf{E} (\textbf{r})=0
\end{equation}

\textbf{Si potrebbe risolvere l'equazione di Helmholtz con le appropriate condizioni al contorno, ma non è facile quindi si deve cercare una seconda via.}

Analizziamo ad esempio un fronte d'onda piana che si propaga attraverso un apertura:

%immagine%
 se usassimo ancora l'ottica geometrica troveremmo che passano solo i raggi centrali mentre gli altri no.


Se l'apertura è molto grande rispetto alla lunghezza d'onda le deflessioni sono meno evidenti e il fronte d'onda è ancora praticamente piano. se però riduciamo l'apertura questa approssimazione non è più accettabile.

Andremo a studiare il sistema per aperture abbastanza piccole da avere diffrazione evidente, ma sufficientemente grande da non variare la polarizzazione del campo entrante lasciando l'equazione di Helmholtz scalare(NON FARLA DIVENTARE VETTORIALE)

\section*{Ottica Rifrattiva in Regime Scalare}
Ipotizziamo di essere nel vuoto:
\begin{equation}
\varepsilon = \varepsilon_0
\end{equation}
chiamo il campo elettrico U(P) anziché \textbf{E} (\textbf{r}) 		(Metto in evidenza il fatto di essere scalare)

\begin{equation}
\nabla^2 U(P) + \left(\frac{\omega}{c}\right)^2 \frac{2}{\varepsilon _0}U(P)=0
\end{equation}

Nell'ottica diffrattiva studiamo tutto ciò che ha dimensione finita 


%IMMAGINI%

prendiamo un sistema con apertura di forma non definita attraverso la quale passa un onda polarizzata linearmente: come sarà il campo dopo l'apertura?

\textbf{Devo risolvere l'equazione di Helmholtz con le apposite condizioni al contorno. dovrò quindi usare le funzioni di Green} 

%cenni sulle funzioni di green%


a formulare la seguente equazione son Kirchhoff, Rayleigh, Sommerfeld

\begin{equation}
U(P_0) = \frac{1}{i\lambda} \iint_\sigma U(P_1) \frac{e^{i k r_{01}}}{r_{01}} \cos(\textbf{n}, \textbf{r}) dS
\end{equation}

%r dovrebbe avere il pecide 01 ma caffacnuclo%
\begin{equation}
k = \frac{\omega}{c}  = \frac{2\pi}{\lambda}
\end{equation}

Esiste una strada alternativa che risulta solo dopo avere visto la soluzione di Rayleigh-Sommerfeld: \textbf{da un punto di vista matematico ci accorgiamo di avere in mano delle onde sferiche}

Si tratta di un potente artificio matematico che scompone il campo nella sovrapposizione di tante funzioni elementari.

Scriveremo il campo di partenza come somma di un set di onde semplici la cui sovrapposizione alla fine ci garantisce di trovare il campo complessivo



\end{document}